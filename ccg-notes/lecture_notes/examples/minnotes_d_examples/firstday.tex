\begin{quote}\textit{
	``There was a mistranslation in Grothendieck's quote, `the rising sea:' he was actually talking about raising
	an \(X\)-wing fighter out of a swamp using the Force.''
	% the force-alg-geo analogy works surprisingly well
	% it's a little bit more like a religion
	% Kylo Ren is Lurie, ...
}\end{quote}
There are a lot of things that go under the scheme of algebraic geometry, but in this class we're going to use the
slogan ``algebra = geometry;'' we'll try to understand algebraic objects in terms of geometry and vice versa.

There are two main bridges between algebra and geometry: to a geometric object we can associate algebra via
functions, and the reverse construction might be less familiar, the notion of a spectrum. This is very similar to
the notion of the spectrum of an operator.

We will follow the textbook of Ravi Vakil,
\href{http://math.stanford.edu/~vakil/216blog/FOAGdec2915public.pdf}{\textit{The Rising Sea}}. There's also a
course website.\footnote{\url{https://www.ma.utexas.edu/users/benzvi/teaching/alggeom/syllabus.html}.} The
prerequisites will include some commutative algebra, but not too much category theory; some people in the class
might be bored. Though we're not going to assume much about algebraic sets, basic algebraic geometry, etc., it will
be helpful to have seen it.

Let's start. Suppose \(X\) is a space; then, there's generally a notion of \(\C\)-valued functions on it, and this
space might be \(F(X)\). For example, if \(X\) is a smooth manifold, we have \(C^\infty(X)\), and if \(X\) is a
complex manifold, we have the holomorphic functions \(\operatorname{Hol}(X)\).\footnote{The best examples here are
Riemann surfaces; when the professor imagines a ``typical'' or example algebraic variety, he sees a Riemann
surface.} Another category of good examples is \term{algebraic sets}, \(X\subset\C^n\) that is given by the common
zero set of a bunch of polynomials: \(X = \{f_1(x) =\dotsb= f_k(x) = 0\}\) for some
\(f_1,\dots,f_k\in\C[x_1,\dots,x_n]\). These have a natural notion of function, \term{polynomial functions}, which
are polynomials in \(\C[x_1,\dots,x_n]\) restricted to \(X\), If \(I(X)\) is the functions vanishing on \(X\), then
these functions are given by \(\C[x_1,\dots,x_n]/I\).

The point is, on all of our spaces, the functions have a natural ring structure.\footnote{In this class, all rings
will be commutative and have a \(1\). Ring homomorphisms will send \(1\) to \(1\).} In fact, there's more: the
constant functions are a map \(\C\to F(X)\), and since \(\C\) is a field, this map is injective. This means
\(F(X)\) is a \(\C\)-algebra, i.e.\ it is a \(\C\)-vector space with a commutative, \(\C\)-linear multiplication.

One of the things Grothendieck emphasized is that one should never look at a space (or an anything) on its own, but
consider it along with maps between spaces. For example, given a map \(\pi:X\to Y\) of spaces, we always have a
\term{pullback} homomorphism \(\pi^*: F(Y)\to F(X)\): if \(f:Y\to\C\), then its pullback is \(\pi^*y(x) =
y(\pi(x))\). This tells us that we have a \term{functor} from spaces to commutative rings.
\subsection*{Categories and Functors.}
This is all done in Vakil's book, but in case you haven't encountered any categories in the streets, let's revisit
them.
\begin{defn}
A \term{category} \(\fC\) consists of a set\footnote{This is wrong. But if you already know that, you know that
worrying about set-theoretic difficulties is a major distraction here, and not necessary for what we're doing, so
we're not going to worry about it.} of \term{objects} \(\operatorname{Ob}\fC\); if \(X\in\operatorname{Ob}\fC\), we
just say \(X\in\fC\). We also have for every \(X,Y\in\fC\) the set \(\Hom_\fC(X,Y)\) of \term{morphisms}. For every
\(X,Y,Z\in\fC\), there's a \term{composition map} \(\Hom_\fC(X,Y)\times\Hom_\fC(Y,Z)\to\Hom_\fC(Y,Z)\) and a unit
\(1_X\in\Hom_\fC(X,X) = \operatorname{End}_\fC(X)\) satisfying a bunch of axioms that make this behave like
associative function composition.
\end{defn}
To be precise, we want categories to behave like monoids, for which the product is associative and unital. In fact,
a category with one object is a monoid. Thus, we want morphisms of categories to act like morphisms of monoids:
they should send composition to composition.
\begin{defn}
A \term{functor} \(F:\fC\to\fD\) is a function \(F:\operatorname{Ob}\fC\to\operatorname{Ob}\fD\) with an
induced map on the morphisms:
\begin{itemize}
	\item If the map acts as \(\Hom_\fC(X,Y)\to\Hom_{\fD}(F(X),F(Y))\), \(F\) is called a \term{covariant}
	functor.
	\item If it sends \(\Hom_\fC(X,Y)\to\Hom_{\fD}(F(Y),F(X))\), then \(F\) is \term{contravariant}.
\end{itemize}
\end{defn}
When we say ``functor,'' we always mean a covariant functor, and here's the reason. Recall that for any monoid
\(A\) there's the \term{opposite monoid} \(A^{\text{op}}\) which has the same set, but reversed multiplication:
\(f\cdot_{\text{op}} g = g\cdot f\). Similarly, given a category \(\fC\), there's an \term{opposite category}
\(\fC\op\) with the same objects, but \(\Hom_{\fC\op}(X,Y) = \Hom_{\fC}(Y,X)\). Then, a contravariant functor
\(\fC\to\fD\) is really a covariant functor \(\fC\op\to\fD\). Hence, in this class, we'll just refer to
functors, with opposite categories where needed.
\begin{ex}
Show that a functor \(\fC\op\to\fD\) induces a functor \(\fC\to\fD\op\).
\end{ex}

When presented a category, you should always ask what the morphisms are; on the other hand, if someone tells you
``the category of smooth manifolds,'' they probably mean that the morphisms are smooth functions.

Now, we see that pullback is a functor \(F:\mathsf{Spaces}\to\mathsf{Ring}\op\). One of the major goals of this
class is to define a category of spaces on which this functor is an equivalence. This might not make sense,
\emph{yet}. This is the seed of ``algebra = geometry.''
\begin{defn} % TODO: pretty functor diagram!
Let \(F,G:\fC\rightrightarrows\fD\) be functors. A \term{natural transformation} \(\eta:F\Rightarrow G\) is a
collection of maps: for every \(X\in\fC\), there's a map \(\eta_X:F(X)\to G(X)\) satisfying a consistency
condition: for every \(f:X\to Y\) in \(\fC\), there's a commutative diagram
\[\xymatrix{
	F(X)\ar[r]^{\eta_X}\ar[d]_{F(f)} & G(X)\ar[d]^{G(f)}\\
	F(Y)\ar[r]^{\eta_Y} & G(Y)
}\]
\end{defn}
That is, a natural transformation relates the objects and the morphisms, and reflects the structure of the
category.
\begin{defn}
A natural transformation \(\eta\) is a \term{natural isomorphism} if for every \(X\in\fC\), the induced
\(\eta_X\in\Hom_{\fD}(F(X),G(X))\) is an isomorphism.
\end{defn}
This is equivalent to having a natural inverse to \(\eta\).

So one might ask, what is the notion for which two categories are ``the same?'' One might naïvely suggest two
functors whose composition is the identity functor, but this is bad. The set of objects isn't very useful: it
doesn't capture the structure of the category. In general, asking for equality of objects is worse than asking for
isomorphism of objects. Here's the right notion of sameness.
\begin{defn}
Let \(\fC\) and \(\fD\) be categories. Then, a functor \(F:\fC\to\fD\) is an \term{equivalence of
categories} if there's a functor \(G:\fD\to\fC\) such that there are natural isomorphisms \(FG\to\id_{\mathsf
D}\) and \(GF\to\id_{\fC}\).
\end{defn}
This is a very useful notion, and as such it will be useful to see an equivalence that is not an isomorphism.
\begin{ex}
Let \(k\) be a field, and let \(\fD = \mathsf{fdVect}_k\), the category of finite-dimensional vector spaces
and linear maps, and let \(\fC\) be the category whose objects are \(\Z_{\ge 0}\), the natural numbers, with an
object denoted \(\ang n\), and with \(\Hom(\ang n,\ang m) = \Mat_{m\times n}\). This is a category with composition
given by matrix multiplication.

Let \(F:\fC\to\fD\) send \(\ang n\mapsto k^n\), and with the standard realization of matrices as linear maps.
Show that \(F\) is an equivalence of categories.
\end{ex}
This category \(\fC\) has only some vector spaces, but for those spaces, it has all of the morphisms.
\begin{defn}
Let \(F:\fC\to\fD\) be a functor.
\begin{itemize}
	\item \(F\) is \term{faithful} if all of the maps \(\Hom_\fC(X,Y)\inj \Hom_\fD(F(X),F(Y))\) are injective.
	\item \(F\) is \term{fully faithful} if all of these maps are isomorphsism.
	\item \(F\) is \term{essentially surjective} if every \(X\in\fD\) is isomorphic to \(F(Z)\) for some
	\(Z\in\fC\).
\end{itemize}
\end{defn}
The following theorem will also be a useful tool.
\begin{thm}
A functor \(F:\fC\to\fD\) is an equivalence iff it is fully faithful and essentially surjective.
\end{thm}
So, to restate, we want a category of spaces that is the opposite category to the category of rings; this is what
Grothendieck had in mind. In fact, let's peek a few weeks ahead and make a curious definition:
\begin{defn}
The \term{category of affine schemes} is \(\mathsf{Rings}\op\).
\end{defn}
Of course, we'll make these into actual geometric objects, but categorically, this is all that we need.

Recall that if \(f:M\to N\) is a set-theoretic map of manifolds, then \(f\) is smooth iff its pullback sends
\(C^\infty\) functions on \(N\) to \(C^\infty\) functions on \(M\). The first step in this direction is the
following theorem, sometimes called \term{Gelfand duality}.
\begin{thm}[Gelfand-Naimark]
The functor \(X\mapsto C^0(X)\) (the ring of continuous functions) defines an equivalence between the category of
compact Hausdorff spaces and the (opposite) category of commutative \(C^*\)-algebras.
\end{thm}
This is an algebro-geometric result: it identifies a category of spaces with the opposite category of a category of
algebraic objects.

However, we need to think harder than Gelfand duality in terms of compact, complex manifolds or in terms of
algebraic spaces: for example, for \(X = \C\P^1\), \(\operatorname{Hol}(X) = \C\): the only holomorphic functions
are constant. The issue is that there are no partitions of unity in the holomorphic or algebraic world. This means
we'll need to keep track of local data too, which will lead into the next few lectures' discussions on \term{sheaf
theory}.

Returning to the example of algebraic sets, suppose \(X\) and \(Y\) are algebraic sets. What is the set of their
morphisms? We decided the ring of functions was the polynomial functions \(Y\to\C\), so we want maps \(X\to Y\) to
be those whose pullbacks send polynomial functions to polynomial functions. To be precise, the \term{ideal of
\(X\)} is \(I(X) = \{f\in\C[x_1,\dots,x_n]\mid f|_X = 0\}\), defining a map \(I\) from algebraic subsets of
\(\C^n)\) to ideals in \(\C[x_1,\dots,x_n]\). There's also a reverse map \(V\),\footnote{\(V\) stands for
``vanishing,'' ``variety,'' or maybe ``vendetta.''} sending an ideal \(I\) to \(V(I) = \{x\in\C^n\mid f(x) =
0\text{ for all } f\in I\}\). From classical commutative algebra, it's a fact that this is finitely generated, so
it's the vanishing locus of a finite number of polynomials, and therefore in fact an algebraic set.

The dictionary between algebraic sets and ideals of \(\C[x_1,\dots,x_n]\) is one of many versions of the
Nullstellensatz (more or less German for the ``zero locus theorem''): if \(J\) is an ideal, \(I(V(J)) = \sqrt J\),
its radical.
\begin{defn}
Let \(R\) be a ring and \(J\subset R\) be an ideal. Then, the \term{radical} of \(J\) is \(\sqrt J = \{r\in R\mid
r^n\in J\text{ for some } n > 0\}\). One says that \(J\) is \term{radical} if \(J = \sqrt J\).
\end{defn}
What this says is that \(J\) is radical iff \(R/J\) has no nonzero nilpotents.\footnote{Recall that if \(R\) is a
ring, an \(r\in R\) is \term{nilpotent} if \(r^n = 0\) for some \(n\).} Why are these kinds of ideals relevant? If
\(X\subset\C^n\) and \(f\) vanishes on \(X\), then so does \(f^n\) for all \(n\). That is, radicals encode the
geometric property of vanishing, which is why \(I(X)\) is a radical ideal.

This is an outline of what classical algebraic geometry studies: it starts by defining algebraic subsets, and
establishing a bijection between algebraic subsets of \(\C^n\) and radical ideals of \(\C[x_1,\dots,x_n]\). This
isn't yet an equivalence of categories. Radical ideals correspond to finitely generated \(\C\)-algebras with no
(nonzero) nilpotents: an ideal \(I\) corresponds to the \(\C\)-algebra \(\C[x_1,\dots,x_n]/I\).

This is all what the course is \emph{not} about; we're going to replace the category of finitely generated,
nilpotent-free \(\C\)-algebras with the category of \emph{all} rings, but we want to keep some of the same
intuition. This involves generalizing in a few directions at once, but we'll try to write down a dictionary; the
defining principle is to identify spaces \(X\) with rings \(R = F(X)\), their ring of functions.

A point \(x\in X\) is a map \(i_x:x\to X\), so we get a pullback \(i_x^*:F(X)\to\C\) given by evaluation at \(x\).
Let \(\m_x = \ker(i_x^*)\); since \(\C\) is a field, this is a maximal ideal.\footnote{Recall that an ideal
\(I\subset R\) is \term{maximal} iff \(R/I\) is a field. This is about the level of commutative algebra that we'll
be assuming.} If \(k\) is a field and \(R\) is a \(k\)-algebra, then \(R/I\) is also a \(k\)-algebra, so in
particular if \(I\) is maximal, then \(k\inj R/I\) is a map of fields, and therefore a field extension.  Thus, if
\(k\) is algebraically closed (e.g.\ we're studying \(\C\)) and \(R\) is a finitely generated \(k\)-algebra, then
maximal ideals of \(R\) are in bijection with homomorphisms \(R\to k\).

Thus, given a ring \(R\), we'll associate a set \(\MSpec(R)\), the set of maximal ideals of \(R\), such that \(R\)
should be its ring of functions. To do this, we'll say that an \(r\in R\) is a ``function'' on \(\MSpec(R)\) by
acting on an \(\m_x\subset R\) as \(r\bmod\m_x\). This is a ``number,'' since it's in a field, but the notion may
be different at every point in \(\MSpec(R)\)! For example, if \(R = \Z\), then \(\MSpec(\Z)\) is the set of primes,
and \(n\in\Z\) is a function which at \(2\) is \(n\bmod 2\), at \(3\) is \(n\bmod 3\), and so on.

A perhaps nicer example is when \(R = \R[x]\), which has maximal ideals \((x-t)\) for all \(t\in\R\). Here,
evaluation sends \(f(x)\mapsto f(x)\bmod (x-t) = f(t)\). That is, this is really evaluation, and here the quotient
field is \(\R\). So these look like good old real-valued functions, but these aren't all the maximal ideals:
\((x^2+1)\) is also a maximal ideal, and \(\R[x]/(x^2+1) \cong \C\). Then, we do get a kind of evaluation again,
but we have to identify points and their complex conjugates.

So we'll have to find a good notion of geometry which generalizes from \(\C\)-algebras to \(k\)-algebras for any
field \(k\), to any commutative rings. We'll also have to think about nilpotents: we threw them away by thinking
about zero sets, but they play a huge role in ring theory.
