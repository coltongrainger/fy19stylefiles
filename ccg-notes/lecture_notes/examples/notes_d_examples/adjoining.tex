One of the most important ways to create fields is to adjoin elements. For example, suppose we have $\Q$ already,
and we would like to solve $x^2 - 2 = 0$. Of course, we can't do that, but we can hope for a solution in an
extension field. Certainly, there are two solutions in $\R$, but is there a ``minimal'' extension $\Q\inj K$ such
that $K$ has a solution to $x^2 - 2$? Creating $K$ from $\Q$ is often called ``adjoining a root of $2$'' or
``adjoining a solution to $x^2 - 2 = 0$.'' The minimal such $K$ is $\Q(\sqrt 2) = \qty{a+b\sqrt 2\mid a,b\in\Q}$.
Thus, we care about field extensions in order to understand roots of polynomials: questions about polynomials can
be turned into questions about field extensions, which we're going to develop methods to solve.

In this example, we knew already knew of an extension of $\Q$ where $x^2 - 2$ had a root, which was helpful. But we
don't know that in general: what if we replaced $\Q$ with $\F_5$? Thankfully, we'll prove Theorem~\ref{rootinext},
which says we can abstractly create an extension to adjoin a root of a polynomial over any field. Then, we'll see
this produces the same answer as concretely producing an extension inside of a larger field, and that if $f$ is an
irreducible polynomial, any two roots of $f$ are ``algebraically indistinguishable,'' in that there extensions
given by adjoining those roots are isomorphic.

Soon, we will develop the algebraic closure $\overline k$ of a field $k$, over which every polynomial has a root.
Knowing this, one might ask why it's worth developing smaller extensions. Not only do we need these results to
develop the algebraic closure, but it's also much easier to do computations in a smaller field extension.

First, though, let's generalize what we did to turn $\Q$ into $\Q(\sqrt 2)$. For the rest of this section, $f$
denotes a nonconstant, irreducible polynomial.
\begin{defn}
\label{fieldgen}
Let $k\inj K$ be a field extension and $a_1,a_2,\dotsc\subset K$. Then, the \term{field generated by}
$a_1,a_2,\dotsc$, denoted $k(a_1,a_2,\dotsc)$, is the smallest subfield of $K$ containing $k$ as well as $a_1$,
$a_2$, and so on.
\end{defn}
In particular, $k(a_1,a_2,\dotsc)$ is an extension of $k$, and embeds into $K$. We've already seen that $\sqrt
2$ and $i$ exist as complex numbers, so we can extend $\Q$ in this way to obtain $\Q(\sqrt 2)$ or $\Q(i)$ or even
$\Q(\sqrt 2,i)$.
\begin{thm}
\label{rootinext}
Let $f\in k[x]$ be a nonconstant, irreducible polynomial. Then, there's an extension $K/k$ such that $f$ has a root
in $K$.
\end{thm}
Algebraically, what does it actually mean that $f$ has a root in $K$, given that $f$ is only defined over $K$? The
extension $i:k\inj K$ induces a map $i_*:k[x]\inj K[x]$ by applying $i$ to each coefficient of a polynomial.
Technically, one says that $i_*f$ has a root in $K$, but there's rarely if ever a need to keep the two separate, so
we identify a polynomial with its image over an extension field.

Another way of thinking of this is that a polynomial defines a function: $x^2+1$, for example, is a function
$\Q\to\Q$ that doesn't vanish. However, we can define the same function with the same coefficients on $\Q(i)$, and
there it vanishes for $x = \pm i$.
\begin{proof}
Since $f$ is nonconstant and irreducible, the ideal $(f)\subset k[x]$ is maximal, and therefore $K = k[x]/(f)$ is a
field. Let $j:k\inj k[x]$ send $a\in k$ to the constant polynomial $a$, and $\pi:k[x]\surj K$ be the canonical
projection, taking everything mod $(f)$. Then, $\pi\circ j: k\to K$ is a field homomorphism, so it's an extension.

Finally, we need to produce a root. Consider $\alpha = \pi(x)\in K$. Since $\pi$ is a ring homomorphism, it
commutes with polynomials, and therefore $p(\alpha) = \pi(p(x))$, and this is $p(x)\bmod p(x) = 0$. Hence, $\alpha$
is a root of $f$.
\end{proof}
% This is a very useful way of producing field extensions, as we'll see later. In particular, it generalizes
% Example~\ref{fieldextexm}\eqref{Gaussrat}: $\Q(i)$ is isomorphic to $\Q[x]/(x^2+1)$, and similarly $\Q(\sqrt 3)
% \cong \Q[x]/(x^2-3)$. Similarly, $\C \cong \R[x]/(x^2+1)$ under the isomorphism $a+bi\mapsto a+bx$. We could use
% this to \emph{define} the complex numbers concretely: rather than supposing that we had a square root of $-1$,
% we've constructed the field $\R[x]/(x^2+1)$, where $x^2 = -1$. In the same way, any polynomial in $\Q[x]$ has a
% root in an extension of $\Q$, which motivates the existence of algebraic numbers.
We still need to show that the extension $K$ constructed in the proof of Theorem~\ref{rootinext} is the
``smallest,'' and relate it to Definition~\ref{fieldgen}. Fortunately, we can do both things at once: in
Theorem~\ref{extr_intr_adj} below, we show that for any extension $K$ of $k$ containing a root $a$ of $f$,
$k(a)\cong k[x]/(f)$ as extensions of $k$. That is, $k[x]/(f)$ embeds into any extension of $k$ containing a root
of $f$ as the subfield that root generates: the extrinsic and intrinsic notions of adjoining an element agree, and
are minimal.
\begin{aside}
Theorem~\ref{extr_intr_adj} can be interpreted in terms of a universal property: given a field extension $k\inj K$
and a root $a\in K$ of $f$, there is a unique map $k[x]/(f)\inj K$ sending $x\mapsto a$. Equivalently, $k[x]/(f)$
is initial in the category of pairs $(K,a)$, where $K/k$ is an extension and $a\in K$ is a root of $f$. This
guarantees that the minimal extension is unique up to unique isomorphism if we can construct something satisfying
its universal property, and this is exactly what Theorems~\ref{rootinext} and \ref{extr_intr_adj} do.

The poset version of this statement might be easier to digest. Inside the poset of field extensions of $k$ (where
$K\le L$ if there's an embedding $K\inj L$), we have a sub-poset $A$ of extensions of $K$ containing a root of $f$.
Theorem~\ref{extr_intr_adj} tells us that $k[x]/(f)$ is the minimal element of this poset.
% TODO: not technically a poset, b/c not exactly a set.
\end{aside}

\begin{thm}
\label{extr_intr_adj}
Let $f\in k[x]$ be a nonconstant, irreducible polynomial and $K/k$ be a field extension containing a root $a$ of
$f$. Then, the assignment $x\mapsto a$ extends to an isomorphism $k[x]/(f)\cong k(a)$ over $k$.
\end{thm}
\begin{proof}
There exists a unique ring homomorphism $\vp:k[x]\to k(a)$ sending $x\mapsto a$ and a constant polynomial
$\lambda\in k$ to $\lambda\in k\inj k(a)$. Its kernel is the ideal of polynomials $p\in k[x]$ such that $p(a) = 0$,
so $f\in\ker(\vp)$ and thus $(f)\subseteq\ker(\vp)$. Thus, $\vp$ factors through the quotient, defining a ring
homomorphism $\widetilde\vp: k[x]/(f)\to k(a)$. Since both $k[x]/(f)$ and $k(a)$ are fields (the former because $f$
is irreducible), then $\widetilde\vp$ is injective; since $k$ and $a$ are both in $\Im(\widetilde\vp)$, then it
must be surjective (since they generate $k(a)$ as a ring inside $K$). Hence, $\widetilde\vp$ is an isomorphism
sending $x\mapsto a$; since $\vp$ sent constant polynomials to their values in $k\inj k(a)$, then so does
$\widetilde\vp$, so it is an isomorphism of field extensions of $k$.
\end{proof}
%eh, I would like to rewrite this someday
Thus, even if we don't have an ambient extension $k\inj K$, we can still talk about adjoining a root $a$ of $f$,
and call the resulting extension $k(a)$.
\begin{ex}
\label{Qdenom}
There is a third notion of ``adjoining an element to a field,'' which is ring-theoretic. Recall that if $R$ is a
subring of a ring $S$, and $a\in S$, then $R[a]$ is the minimal subring of $S$ containing both $R$ and $a$.
Specializing to fields, consider a finite\footnote{Finiteness is not necessary, but some sort of condition is
needed: $k[x]\ne k(x)$ inside $k(x)$.} field extension $k\inj K$ and an $a\in K$. Show that $k[a] = k(a)$.
\end{ex}
This is why you could ``rationalize the denominator'' in your high school algebra classes: an element of, say,
$\Q(\sqrt 2)$, which may have radicals in the denominator, is equal to an element of $\Q[\sqrt 2]$, and everything
in $\Q[\sqrt 2]$ is generated by $\Q$ and $\sqrt 2$ as a ring, meaning we can multiply by $\sqrt 2$, but not
divide: no rationals in the denominator.

Another takeaway of Theorem~\ref{extr_intr_adj} is that, intrinsically, all the roots of an irreducible polynomial
``look the same'' algebraically: if $a$ and $b$ are both roots of $f$ inside some extension field $K$, then
$k(a)\cong k[x]/(f)\cong k(b)$. For example, this has the curious consequence that if $\alpha$ is one of the
complex cube roots of $-2$, then there's no intrinsic way to distinguish the fields $\Q(\sqrt[3]{-2})$ and
$\Q(\alpha)$, even though one consists only of real numbers and the other doesn't.

The takeaway is that some field properties one might find important are actually properties of its embedding inside
of a larger field: what distinguishes these two fields is how they sit inside $\C$. The embedding
$\Q(\sqrt[3]{-2})\inj\C$ factors through $\R\inj\C$, but the embedding $\Q(\alpha)\inj\C$ does not. It's important
to be careful about what's intrinsic versus extrinsic, and for this reason, people often prefer to adjoin elements
abstractly, e.g. ``let $\omega$ denote a cube root of unity and $K = \Q(\omega)$,'' rather than saying ``let $K =
\Q((-1+i\sqrt 3)/2)$,'' to emphasize that the discussion is true for all three $\omega$ satisfying $x^3 - 1 = 0$,
rather than for the specific one chosen.

Formally, instead of saying that the roots of an irreducible polynomial ``look the same,'' one says they're
\term{algebraically indistinguishable}. One useful consequence is that it isomorphisms extend: we'll use the
following lemma several times, mostly as an ingredient in uniqueness results.
\begin{lem}[Extension]
\label{extension_lem}
Let $\vp:K\to L$ be an isomorphism of fields, $f\in K[x]$ be a nonconstant irreducible polynomial, and $g\in L[x]$
be its image under $\vp$. If $a$ is a root of $f$ and $b$ is a root of $g$, there exists an isomorphism
$\widetilde\vp: K(a)\to L(b)$, in the sense that the following diagram commutes:
\begin{equation}
\label{extndiag}
\begin{gathered}
\xymatrix{
	K(a)\ar[r]^\cong_{\widetilde\vp} &L(b)\\
	K\ar@{^(->}[u]\ar[r]^\cong_\vp & L.\ar@{^(->}[u]
}
\end{gathered}
\end{equation}
\end{lem}
\begin{proof}
The isomorphism $\vp$ induces an isomorphism $K[x]\to L[x]$ defined by sending
\[a_nx^n + a_{n-1}x^{n-1} + \dotsb + a_1x + a_0\mapsto \vp(a_n)x^n + \vp(a_{n-1})x^{n-1} + \dotsb + \vp(a_1)x +
\vp(a_0),\]
and this maps $f$ to $g$. Thus, it maps $(f)$ to $(g)$, so it defines an isomorphism $\psi: K[x]/(f)\to
L[x]/(g)$. By Theorem~\ref{extr_intr_adj}, $K[x]/(f)\cong K(a)$ is an isomorphism of extensions of $K$, and
similarly for $L[x]/(g)\cong L(b)$, so these isomorphisms fit together into the following commutative diagram of
field homomorphisms, for which all horizontal arrows are isomorphisms.
\[\xymatrix@C=0.2cm{
	K(a)\ar[rr] && K[x]/(f)\ar[rrr]^\psi &&& L[x]/(g)\ar[rr] && L(b)\\
	& K\ar@{^(->}[ul]\ar@{_(->}[ur]\ar[rrrrr]^\vp &&&&& L\ar@{^(->}[ul]\ar@{_(->}[ur]
}\]
Taking $\widetilde\vp$ to be the composition across the top row reduces this diagram to~\eqref{extndiag} as
desired.
\end{proof}
