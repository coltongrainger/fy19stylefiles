Galois theory involves studying polynomials over a field, which are ubiquitous in algebra and number theory. So we
start with the basics of fields.
\begin{defn}
Recall that a \term{field} $k$ is a commutative ring with $1$ such that $1\ne 0$ and every nonzero element is
invertible, and $1\ne 0$.
\end{defn}
That is: in a field we can add, subtract, multiply, and divide, though as usual we can't divide by zero. A ``field
homomorphism'' (meaning a structure-preserving map) is just a ring homomorphism; we ask to preserve addition and
multiplication, and subtraction and division come for free.
\begin{comp}{ex}{enumerate}
	\item The rational numbers $\Q$, the real numbers $\R$, and the complex numbers $\C$ are all fields.
	\item Let $p$ be prime. Then, $\Z/p$ is a field, called the \term{finite field of order $p$} and denoted
	$\F_p$.
	\begin{proof}
	First, we show $\Z/p$ is an integral domain: if $m\cdot n\equiv 0\bmod p$, then $m\cdot n = xp$ for some $x$,
	so $p$ divides either $m$ or $n$. Thus, either $m$ or $n$ is $0$ in $\Z/p$.

	Now, for any nonzero $m\in\Z/p$, consider the multiplication map $\vp_m\cdot:\Z/p\to\Z/p$ sending $n\mapsto
	mn$. If $a,b\in\Z/p$ are such that $\vp_m(a) = \vp_m(b)$, then $\vp_m(a-b) = m\cdot(a-b) = 0$. Since $m$ is
	nonzero and $\Z/p$ is an integral domain, then $a = b$, and so $\vp_m$ is injective. Since $\Z/p$ is finite, an
	injective map from $\Z/p$ to itself is a bijection. Thus, there's a unique $n\in\Z/p$ such that $\vp_m(n) = 1$,
	or $1 = mn$. That is, every nonzero element is invertible.
	\end{proof}
	\item If $k$ is any field, we can form the \term{field of rational functions} in $k$, denoted $k(x)$, to be
	ratios $p/q$ for polynomials $p,q\in k[x]$ with $q\ne 0$. We'd like them to be ``in lowest terms,'' but this is
	a clunky definition and it's simpler to just say that two rational functions $p/q$ and $p'/q'$ are the same if
	we can cross-multiply: $pq' = p'q$.
\end{comp}
There's not a whole lot we can say about a field $k$ in total generality, without knowing more about it, but we
know $1\in k$, and therefore $1+1\in k$, and $1+1+1\in k$, and so on. These numbers might all be distinct, like for
$\Q$, or might not be, like for $\F_p$.
\begin{defn}
The \term{characteristic} $\chr(k)$ of a field $k$ is the smallest multiple of $1$ that is equal to $0$ in $k$, or
is $0$ if no such multiple exists.
\end{defn}
For example, $\chr(\F_p) = p$, and $\chr(\Q) = \chr(\R) = 0$.
\begin{ex}
Show that for any field $k$, $\chr(k)$ is either $0$ or a prime number.
\end{ex}
The characteristic is an important property of a field: many important things in Galois theory are different in the
characteristic $0$ case and the characteristic $p$ case.

One of the consistent lessons of algebra is to study objects by looking at their homomorphisms. In the case of
fields, morphisms are the setting of Galois theory.
\begin{lem}
Let $\vp:k\to K$ be a homomorphism of fields. Then, $\vp$ is injective.
\end{lem}
\begin{proof}
The kernel $\ker(\vp)\subset k$ is an ideal of $k$. Since $k$ is a field, its only ideals are $0$ and $k$ itself.
If $\ker(\vp) = 0$, then $\vp$ is injective, as desired; if $\ker(\vp) = k$, then $\vp(1) = 0$, which is
impossible, because ring morphisms must send $1$ to $1$.
\end{proof}
This is very different than for other kinds of algebraic objects: no interesting kernels and no interesting
quotients.
\begin{defn}
If $k$ is a field, a \term{field extension} of $k$ is a homomorphism $i:k\inj L$, often written
$L/k$.\footnote{This notation looks like a quotient, but since we will never take the quotient by a field
extension, this is not ambiguous. We will take quotients of rings, however.}
\end{defn}
Galois theory is the study of field extensions and relations between them.
\begin{defn}
Let $k$ be a field and $i_K:k\inj K$ and $i_L:k\inj L$ be field extensions. An \term{embedding}
(sometimes said to be an \term{embedding over $k$}) is a field homomorphism $j:K\inj L$ such that the following
diagram commutes:
\[\xymatrix{
	K\ar[rr]^j && L\\
   & k.\ar[ul]_{i_K}\ar[ur]^{i_L} &
}\]
That is, $i_L = j\circ i_K$.
\end{defn}
Embeddings keep track of how one field lies as a subfield of another.
\begin{defn}
If $k$ is a field, its \term{prime subfield} is the subfield of $k$ generated by $1$.
\end{defn}
That is, the prime subfield is the smallest field containing $1$ inside $k$, meaning it must contain $1+1$,
$1+1+1$, and so forth. If $\chr(k) = 0$, this generates a copy of $\Z$ inside $k$, and therefore $\Q$ also, since
we can invert all nonzero elements of $\Z$. That is, if $\chr(k) = 0$, then the prime subfield of $k$ is $\Q$.
In the same way, if $\chr(k) = p$, then its prime subfield is $\F_p$.
\begin{coro}
If $K/k$ is a field extension, then $\chr(k) = \chr(K)$.
\end{coro}
This is because the prime subfield of $K$ contains the prime subfield of $k$.
\begin{comp}{ex}{enumerate}
\label{fieldextex}
	\item Since $\Q$ is a subfield of $\R$, the inclusion $\Q\inj\R$ is a field extension. In the same way,
	$\R\inj\C$ is a field extension; this fixes $\Q$, so it's an embedding over $\Q$.
	\item\label{Gaussrat} The \term{field of Gaussian rationals} is $\Q(i) = \qty{a+bi\mid a,b\in\Q}$.
	\begin{ex}
	Show that $\Q(i)$ is a field.
	\end{ex}
	If $a\in\Q$, $a = a+0i\in\Q(i)$, so $\Q\inj\Q(i)$ is another example of a field extension. One can form a
	similar definition for, e.g.\ $\Q(\sqrt 3)$ or $\Q(\sqrt{-2})$, but we'll soon define something much more
	general.
\end{comp}
\begin{aside}[Categorical language in field theory]
The modern formulation of abstract algebra tends to use categorical language, defining categories of algebraic
objects such that useful constructions satisfy universal properties. However, this is uncommon for field
theory, as the category of fields is poorly behaved: few products exist (e.g.\ the ring $\Q\times\Q$ is not a
field), there are no initial or final objects, and the category is disconnected, since a map of fields must
preserve the characteristic. Specializing to the category of fields of a given characteristic fixes some, but not
all, of these problems.

Nonetheless, there are a few places where words from category theory will simplify things, and I'll try to mention
them as they happen. Since they may require knowledge beyond what I assume for these notes, they will also be
marked as asides.

Fixing a base field $k$, we can define the \term{category of field extensions} $\FExt_k$ to be the category whose
objects are field extensions $K/k$ and whose morphisms are embeddings $K\inj L$ over $k$.

{\color{red}TODO}: talk about the poset? Eventually will be a lattice.
\end{aside}
\begin{lem}
If $i:k\inj K$ is a field extension, then $K$ is a $k$-vector space.
\end{lem}
\begin{proof}
We need to define an action of $k$ on $K$, which will just be multiplication: if $\lambda\in k$ and $x\in K$, let
$\lambda\cdot x = i(\lambda)x$. The field axioms of $K$ imply (since $k$ is realized as a subfield of $K$) that
multiplication satiafies the axioms for a vector space.
\end{proof}
\begin{defn}
\label{finite_ext}
If $K/k$ is a field extension, its \term{degree}, written $[K:k]$, is the dimension of $K$ as a $k$-vector space.
If this is a finite number, $K/k$ is said to be a \term{finite extension}; otherwise, it's an \term{infinite
extension}.
\end{defn}
This is analogous to the index of a subgroup $[G:H]$. For example, $[\C:\R] = 2$, so $\C/\R$ is a finite extension,
but $\R/\Q$ is an infinite extension.

%The first important reason we care about field extensions is to discuss roots of polynomials. For example, $x^2-2$
%doesn't have a root in $\Q$, but it has $2$ roots in $\Q(\sqrt 2)$. This enables questions about polynomials to be
%turned into questions about field extensions, which can be solved with algebraic methods.
% \begin{thm}
% \label{rootinext}
% Let $f\in k[x]$ be a nonconstant polynomial. Then, there's an extension $K/k$ such that $f$ has a root in $K$.
% \end{thm}
% Algebraically, what does it actually mean that $f$ has a root in $K$, given that $f$ is only defined over $K$? The
% extension $i:k\inj K$ induces a map $i_*:k[x]\inj K[x]$ by applying $i$ to each coefficient of a polynomial.
% Technically, one says that $i_*f$ has a root in $K$, but there's rarely if ever a need to keep the two separate, so
% we identify a polynomial with its image over an extension field.
% 
% Another way of thinking of this is that a polynomial defines a function: $x^2+1$, for example, is a function
% $\Q\to\Q$ that doesn't vanish. However, we can define the same function with the same coefficients on $\Q(i)$, and
% there it vanishes for $x = \pm i$.
% \begin{proof}
% Without loss of generality, assume $f$ is irreducible: if not, replace $f$ with one of its irreducible factors. A
% field containing a root of a factor of $f$ must then contain a root of $f$.
% 
% Since $f$ is nonconstant and irreducible, the ideal $(f)\subset k[x]$ is maximal, and therefore $K = k[x]/(f)$ is a
% field. Let $j:k\inj k[x]$ send $a\in k$ to the constant polynomial $a$, and $\pi:k[x]\surj K$ be the canonical
% projection, taking everything mod $(f)$. Then, $\pi\circ j: k\to K$ is a field homomorphism, so it's an extension.
% 
% Finally, we need to produce a root. Consider $\alpha = \pi(x)\in K$. Since $\pi$ is a ring homomorphism, it
% commutes with polynomials, and therefore $p(\alpha) = \pi(p(x))$, and this is $p(x)\bmod p(x) = 0$. Hence, $\alpha$
% is a root of $f$.
% \end{proof}
% This is a very useful way of producing field extensions, as we'll see later. In particular, it generalizes
% Example~\ref{fieldextex}\eqref{Gaussrat}: $\Q(i)$ is isomorphic to $\Q[x]/(x^2+1)$, and similarly $\Q(\sqrt 3)
% \cong \Q[x]/(x^2-3)$. Similarly, $\C \cong \R[x]/(x^2+1)$ under the isomorphism $a+bi\mapsto a+bx$. We could use
% this to \emph{define} the complex numbers concretely: rather than supposing that we had a square root of $-1$,
% we've constructed the field $\R[x]/(x^2+1)$, where $x^2 = -1$. In the same way, any polynomial in $\Q[x]$ has a
% root in an extension of $\Q$, which motivates the existence of algebraic numbers.

{\color{red}TODO}: treat this in a unified way with adjoining an element; explain what it means to ``adjoin a
square root of $2$,'' and that this is algebraically indistinguishable (maybe this should be another section, and
then irreducibility criteria is a third section).

\subsection*{Irreducibility criteria}
The key of Theorem~\ref{rootinext} is that $f$ is irreducible; if we take $k[x]/(f)$ for a reducible $f$,
$(f)$ isn't even prime, so the resulting quotient isn't a field, or even an integral domain! So in practice, we
need to know when a polynomial is irreducible. Here are a few criteria.
\begin{lem}
\label{cubirred}
If $f\in k[x]$ has degree $2$ or $3$, then $f$ is irreducible iff it has no roots.
\end{lem}
Of course, the hypothesis is necessary: $(x+2)^2\in\Q[x]$ is reducible, but has no roots.
\begin{thm}[Rational root theorem]
{\color{red}TODO}
\end{thm}
\begin{lem}[Gauss' lemma]
\label{gausslem}
Let $R$ be a UFD and $k$ be its field of fractions. If $f\in R[x]$, then if $f$ is reducible in $k[x]$, then it's
reducible in $R[x]$.
\end{lem}
This is most often used when $R = \Z$, so $k = \Q$: then, it says that \emph{an $f\in\Z[x]$ is irreducible iff it's
irreducible in $\Q[x]$}.
\begin{prop}
Let $R$ be an integral domain and $I\subset R$ be a proper ideal. If $f\in R[x]$ and $f\bmod I$ is irreducible in
$(R/I)[x]$, then $f$ is irreducible over $R$.\footnote{The converse is untrue: $x^4+1$ is irreducible over $\Z$,
but reducible modulo every prime.}
\end{prop}
This is most often used when $R = \Z$ and $I = (p)$ for some prime $p$.
\begin{prop}[Eisenstein's criterion]
Let $R$ be an integral domain and $\p\subset R$ be a prime ideal. If $f(x) = x^n + a_{n-1}x^{n-1} + \dotsb + a_1x +
a_0$ is such that $a_{n-1},\dotsc,a_0\in\p$ and $a_0\not\in\p^2$, then $f$ is irreducible in $R[x]$.
\end{prop}
This is generally used for $R = \Z$ and $\p = (p)$ for a prime number $p$. In this case, it says the following.
\begin{coro}
Let $f\in\Z[x]$ be given by $f(x) = x^2 + a_{n-1}x^{n-1} + \dotsc + a_1x + a_0$, and suppose there's a prime number
$p\in\Z$ such that $p\mid a_i$ for $i = 0,\dotsc,n-1$, and $p^2\nmid a_0$. Then, $f$ is irreducible in $\Z[x]$.
\end{coro}
By Lemma~\ref{gausslem}, this also implies $f$ is irreducible in $\Q[x]$. The most common application of
Eisenstein's criterion is to show that a given polynomial in $\Z[x]$ is irreducible over $\Q$.

{\color{red}TODO} fill in these proofs.

Doing a few exercises using these results will probably be more helpful than reading their proofs.
\begin{proof}[Proof of Lemma~\ref{cubirred}]
If $f$ is reducible, then $f = gh$, where $g$ and $h$ are polynomials of degree at least $1$. Since $\deg(g) +
\deg(h)\le 3$, this means at least one of $g$ or $h$ has degree exactly $1$. Without loss of generality, assume
it's $g$, so $g(x) = ax + b$ for $a,b\in k$ and $a\ne 0$; then, $-b/a$ is a root of $g$, and therefore of $f$.

Conversely, if $f$ has a root, then it's reducible.
\end{proof}

In the next few sections, we'll develop the theory of a few nice kinds of field extensions.
