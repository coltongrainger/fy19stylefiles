\begin{quote}\textit{
	``I didn't give it a good notation because I didn't like it.''
}\end{quote}
Recall that last time, we defined sheafification, which can be thought of projecting presheaves onto sheaves in a
particularly nice way. This allows us to forget the difference between sheaves and presheaves, so to speak; we'll
use this to understand colimits of sheaves.
\begin{exm}% would be nice to say what x, i_* are...
First, a quick digression, since we got confused last time, on the espace étalé of a skyscraper sheaf. Directly
from the sheaf axioms, one can show that if \(\sF\) is a \(\fC\)-valued sheaf, then \(\sF(\emptyset)\) is the
terminal object (a point for \(\Set\), \(0\) for \(\Ab\), and so on). This follows from abstract nonsense: the
empty product \(\prod_\emptyset S\) is necessarily the terminal object (there's more to think through here). This
is what motivates the definition of the skyscraper sheaf \(i_*S = i_{x,*}S\) in Example~\ref{skys}. For
simplicity, assume \(x\in X\) is a closed point.

Now, let's construct its espace étalé \(\pi:Y_{i_*S}\to X\); for any \(y\in X\), \(\pi^{-1}(y)\) is the stalk of
\((i_*S)\) at \(y\), which is \(S\) if \(y = x\) or the terminal object \(*\) otherwise. Thus, \(Y_{i_*S}\) is as a
set a copy of \(X\), but with \(S\) over the basepoint \(x\) instead of a single point; then, we glue each of these
points of \(S\) to the rest of \(Y_{i_*S}\) as if they were all \(x\). The result is \(X\) with multiple
basepoints, so to speak, and is not at all Hausdorff. However, as topological spaces, we have a pullback diagram
\[\xymatrix{
	(U\setminus\set x)\times S\ar[r]\ar[d] & U\times S\ar[d]\\
	U\setminus\set x\ar[r] & Y_{i_*S}.
}\]
\end{exm}
We can also use the espace étalé to define sheafification: the sheafification \(\sF\sh\) is just the sheaf of
sections of \(Y_\sF\).
\subsection*{Kernels and Cokernels.}
Before discussing limits and colimits more generally, let's focus on kernels and cokernels. Let \(\vp:\sF\to\sG\)
be a morphism of sheaves of abelian groups on a space \(X\), and for every open \(U\subset X\), define
\((\ker\vp)(U) = \ker(\vp|_U)\). It's easy to check that this is a presheaf, and a little more work to check that
it's a sheaf, too. And this is actually the kernel in \(\Ab_X\), in that it satisfies the universal property:
it fits into the diagram
\[\xymatrix{
	\ker\vp\ar[r]\ar[d] & \sF\ar[d]^\vp\\
	0\ar[r] & \sG,
}\]
and any other sheaf \(\mathscr H\) that fits into the same place in the above diagram has a unique map to
\(\ker\vp\).

Likewise, a morphism in \(\Ab_X\) is injective (meaning a monomorphism) exactly when
\(\vp|_U:\sF(U)\to\sG(U)\) is injective for all open \(U\subset X\).

Cokernels are a little more interesting. The sheaf assigning \(U\mapsto\coker(\vp|_U)\) is a presheaf, and is the
cokernel in the category of presheaves, but it is \emph{not} the cokernel in the category of sheaves; it fails to
satisfy the universal property. This is where some of the interesting nuances of sheaf theory pop up.
\begin{exm}
\label{notcoker}
We'll let \(X = \C\), and let \(\sO\) be the sheaf of holomorphic functions and \(\sO^*\) be the sheaf of
``invertible,'' i.e.\ nonvanishing, holomorphic functions (an abelian group under multiplication). The exponential
map \(f(z)\mapsto e^{f(z)}\) sends holomorphic functions to nonvanishing holomorphic functions, and commutes with
restriction, so it's a morphism \(\exp:\sO\to\sO^*\) in \(\Ab_\C\).

If a function maps to \(1\) in \(\sO^*\), then it must be an integer multiple of \(2\pi i\), so it must be locally
constant, Thus, it's constant on each connected component of the given open set. Thus, \(\ker(\exp) = 2\pi
i\underline\Z\): the constant sheaf, not the constant presheaf. This agrees with what we just learned about
kernels.

Then, \(\Im(\exp)(U)\) is the \(f^*\in\sO^*(U)\) such that \(f = e^{2\pi i g}\) for some \(g\in\sO(U)\). That is,
\(\log f\) must have a well-defined branch on \(U\). In particular, if \(U = \C^*\) and \(f = z\), then
\(f\not\in\Im(\exp(U)\). This is a problem: \(\C^*\) can be covered by simply connected open sets on which the
logarithm exists, but the gluing axiom fails.

However, since \(\exp:\sO\to\sO^*\) is surjective on simply connected open sets, then it's surjective on the level
of stalks, even though it's not surjective as a map of sheaves. In other words, we want the sequence
\[\shortexact[j][\exp]{\underline\Z}{\sO}{\sO^*}{}\]
%\[\xymatrix{
%	0\ar[r] &\underline\Z\ar[r]^j &\sO\ar[r]^\exp & \sO^{*}\ar@{-->}[r] & 0
%}\]
to be a short exact sequence of sheaves, but if we naïvely define the cokernel like the kernel, it isn't. This
means that \emph{to define the sheaf cokernel, we sheafify the presheaf cokernel}. In this case, the sheafification
of the presheaf cokernel \(\coker(j)\) stitches together the stalks, but on stalks \(\exp\) is surjective, so since
a sheaf is completely determined by stalks, this is just \(\sO^*\) again, which jives with the idea of
surjectivity. In the same way, we get that \(\coker(\exp) = 0\), as one would expect.
\end{exm}
In other words, a surjective map of sheaves (categorically, an epimorphism), is surjectivity on stalks, but
\emph{not} surjectivity on all open subsets. Injectivity is equivalent to injectivity on stalks and on open
subsets, though.

Since sheafification preserves colimits, this can be generalized: the colimit of a diagram of sheaves is the
sheafification of the presheaf colimit (which is just the colimit on every open set).
\begin{exm}
This next example is in some sense the same example. Let \(X\) be a smooth manifold, \(\sF\) be the sheaf of
smooth maps to \(S^1\), \(C^\infty\) be the smooth maps to \(\R\) (so just the smooth functions), and
\(\underline\Z\) is the constant sheaf (which is also smooth maps to \(\Z\), since \(\Z\) is discrete); each of
these is a sheaf of abelian groups.

We'd like to understand that \(S^1 = \R/\Z\). This comes from the sequence
\[\xymatrix{
	0\ar[r] & \underline\Z\ar[r] & C^\infty\ar[r] & \sF\ar[r] & 0,
}\]
which is short exact. The injectivity of \(\underline\Z\inj C^\infty\) comes from the fact that every map to \(\Z\)
can be lifted to a smooth map to \(\R\), and surjectivity comes from the fact that germs of functions can be
lifted on a small neighborhood, so it's surjective on stalks. However, there are open subsets where functions can't
be lifted: if \(X = S^1\), then the identity map \(S^1\to S^1\) can't be lifted to a map to \(\R\). Thus, this is
surjective, even though it's not so on the level of open sets.
\end{exm}
\begin{exm}
Our next example will be the de Rham complex. Let \(X\) be a smooth manifold. Let \(\underline\R\) denote the
constant sheaf on \(\R\) (locally constant functions) and \(\Omega^1\) denote the sheaf of one-forms on \(X\). The
exterior derivative gives us an exact sequence
\[\xymatrix{
	0\ar[r] & \underline\R\ar[r] & C^\infty\ar[r]^{\d} & \Omega^1\ar[r]^{\d} & \Omega^2\ar[r]^{\d} &\dots
}\]
However, this is not in general short exact; if \(\Omega_{\text{cl}}^1\) denotes the space of closed one-forms,
then the Poincaré lemma just states that the following sequence is short exact.
\[\xymatrix{
	0\ar[r] & \underline\R\ar[r] & C^\infty\ar[r]^{\d} & \Omega_{\text{cl}}^1\ar[r] & 0
}\]
In other words, even considering something very simple about short exact sequences of sheaves gives us cohomology.
This can be used to define sheaf cohomology, though we won't return to that anytime soon.

In fact, Example~\ref{notcoker} is a special case of this, since \(\d z/z\in\Omega^1(\C^*)\) is a closed form
that's not exact.
\end{exm}
\subsection*{Ringed Space.}
Anyways, we were going to talk about schemes, right? These are not just topological spaces, but ringed spaces:
topological spaces with a notion of a ring of functions.
\begin{defn}
A \term{ringed space} is the data \((X,\sO_X)\), where \(X\) is a topological space and \(\sO_X\) is a sheaf of
rings on \(X\).
\end{defn}
The motivating examples are a topological space with continuous functions to \(\R\) (since these form a ring), or a
smooth manifold with the sheaf \(C^\infty\), or an analytic manifold with \(C^\omega\) (analytic functions).  Thus,
there are definitely different notions of ``function'' on a manifold, but the ringed space structure means knowing
what kinds of functions (geometric structure) is.

We'd also like to know how to evaluate functions on a ringed space. For an arbitrary \(x\in U\) and
\(f\in\sO_X(U)\), it's not clear how to define \(f(x)\); we have stalks, but then what? In each of our examples
(continuous functions, smooth functions, analytic functions, holomorphic functions, etc.), the stalks \(\sO_{X,x}\)
aren't just rings, but local rings,\footnote{Recall that a \term{local ring} is a ring with a unique maximal
ideal.} with the maximal ideal \(\m_x\) of functions which vanish at \(x\). \(\m_x\) is unique, because if
\(f\in\sO_{X,x}\setminus\m_x\), then \(f(x)\ne 0\), so it's nonzero on a neighborhood of \(x\), and therefore
invertible in that subset! Thus, \(f\in\sO_{X,x}^\times\), so \(\m_x\) must be unique.

The point is, evaluating at \(x\) is exactly quotienting by \(\m_x\), producing an element of \(\R\). The sheaves
we care about have local rings for stalks, which is what makes this evaluation work. We'll turn this into a
definition of something much more useful than a ringed space.
\begin{defn}
A \term{locally ringed space} is a ringed space \((X,\sO_X)\) such that for every \(x\in X\), the stalk
\(\sO_{X,x}\) is a local ring.
\end{defn}
Thus, all of our basic examples are locally ringed spaces, and in general, given an \(f\in\sO_X(U)\), we can define
\(V(f) = \set{x\in U: f(x) = 0}\), and this will end up being a closed set.

Schemes are particular examples of locally ringed spaces. We'll have to define how to produce a sheaf of functions,
which we'll probably do next time, but we're almost there. One major takeaway is that schemes behave somewhat like
these examples we already have.

We also need to define morphisms. An isomorphism is evident: \((X,\sO_X)\cong (Y,\sO_Y)\) is the data of a
homeomorphism \(f:X\to Y\) that identifies the sheaves, i.e.\ for all open \(U\subset Y\), there's an isomorphism
\(f_*: \sO_Y(U)\to \sO_X(f^{-1}(U))\).

It's less obvious how to define morphisms in general; clearly, we need a continuous \(f:X\to Y\), and we want to
compare \(\sO_X\) and \(\sO_Y\). Functions pull back (because the preimage of an open set is open); in the examples
we had before, we checked that the pullbacks of continuous (smooth, etc.) functions were continuous (smooth, etc.).
More generally, given an open \(U\subset Y\), we have the two rings \(\sO_Y(U)\) and \(\sO_X(f^{-1}(U))\), and we
want the pullback of functions \(f_*:\sO_Y(U)\to\sO_X(f^{-1}(U))\) to be a ring homomorphism. This is exactly how
we defined the pushforward of a sheaf.
\begin{defn}
A \term{morphism of ringed spaces} is a pair \((f,f^\sharp): (X,\sO_X)\to (Y,\sO_Y)\) in which
\begin{itemize}
	\item \(f:X\to Y\) is continuous, and
	\item \(f^\sharp: \sO_Y\to f_*\sO_X\) is a morphism in \(\Ring_Y\).
\end{itemize}
\end{defn}
That is: for every open subset, we can pull functions back into that open subset. But we can say that more
concisely with the sheaf theory we have developed.

It's worth remembering that nilpotents on affine schemes give us functions that aren't determined by their values
(well, we do have to set up the structure of a locally ringed space first, but we'll get there), so a function
isn't quite a bunch of values at points; it's something that we care to pull back.

This is cool, but we care about ringed spaces. What about these maximal ideals? They tell us what it means for a
function to vanish. Back in the world of smooth functions, if \(\vp(y) = 0\) and \(x\in f^{-1}(y)\), then
\((f^*\vp)(x) = \vp(f(x))\) had better be \(0\) too. This is not preserved by morphisms of ringed spaces (since
evaluation isn't defined for germs of functions on ringed spaces), so we need an additional axiom.

If \((f,f^\sharp)\) is a morphism of ringed spaces, passing to colimits induces a map \(\sO_{Y,y}\to\sO_{X,x}\),
whenever \(f(x) = y\) (this is generally true for a map of sheaves, thanks to the property of colimits). Then, we
want this map to send \(\m_y\to\m_x\). 
\begin{defn}
A morphism \((f,f^\sharp):(X,\sO_X)\to(Y,\sO_Y)\) is a morphism of locally ringed spaces if for every \(x\in X\),
\(y\in Y\) such that \(f(x) = y\), the induced map \(f^\sharp_x: \sO_{Y,y}\to\sO_{X,x}\) maps \(\m_y\) into
\(\m_x\).
\end{defn}
This is actually all the data that we'll need to define schemes: schemes are a full subcategory of locally ringed
spaces; specifically, they are the ones that are locally isomorphic to \((\Spec R, \sO_{\Spec R})\) (as soon as we
define the locally ringed space structure on \(\Spec R\)), i.e.\ there are actual isomorphisms on an open cover.

Does this look weird? It's actually not unfamiliar: a smooth manifold is a locally ringed space that's locally
isomorphic to \((\R^n,C^\infty)\). This encodes a lot of information; in particular, a continuous map of manifolds
is smooth iff it pulls smooth functions back to smooth functions. In the same way, a topological manifold is a
locally ringed space locally isomorphic to \((\R^n,C)\) (the sheaf of continuous functions). All the structure of
an atlas is encapsulated in this notion of locally ringed spaces.

This notion is extremely general. For example, we can define a complex analytic manifold to be a locally ringed
space locally isomorphic to \((U\subset\C^n, \operatorname{Hol})\) (since small discs in \(\C^n\) aren't
necessarily biholomorphic to all of \(\C^n\)). In all of the cases we've seen, though, \(\sO_X(U)\) is always a
subset of set maps \(U\to\R\) (or \(\C\)), and in particular functions are determined by their values. This is
something that will not be true for schemes.

Next time, we will define \(\Spec R\), as a scheme.
