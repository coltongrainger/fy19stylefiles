% this style file is borrowed from Arun Debray <ccg, 2019-04-14> %
% the mistakes are my own %

%
\RequirePackage{fancyhdr}
\RequirePackage{titlesec}
\RequirePackage[dotinlabels]{titletoc}
% after the font has been chosen
\AtBeginDocument{
	\usepackage{microtype}
}
%
% Set colors for hyperref
\hypersetup{
	colorlinks,
	linkcolor={red!50!black},
	citecolor={green!50!black},
	urlcolor={blue!80!black}
}
%
% Custom section titles
% I really want to not depend on titlesec, as it's frustrating to use. However,
% this setup is stable.
\titleformat{\section}[frame]
  {\normalfont}
  {\filright
   \footnotesize
   \enspace \arabic{section}.\enspace}
  {8pt}
  {\Large\bfseries\filcenter}
\titlecontents{section}[1.5em]{}{\contentslabel{2.3em}}{\hspace*{-2.3em}}{\hfill\contentspage}
%
% Ornament to divide different parts of a section without introducing a new subsection
\newcommand{\orbreak}{
\begin{center}
	\adforn{17}\;\(\cdot\)\;\adforn{18}
	\vspace{0.2cm}
\end{center}
}
%
% some other stylistic changes
% \def\qedsymbol{{\small{\ensuremath{\boxtimes}}}}
%
% The user should specify the following information:
%	\topic: what's this document? Neither lecture notes nor a problem set.
%	\institution: where the course was held, e.g. "UT Austin"
%	\coursenum: course dept. and number, e.g. "M364C"
%	\coursename: name of the course, e.g. "Galois Knot Theory"
%	\semester: when the course happened, e.g. "Fall 2016"
%	\author: who took notes, e.g. "Ted Geisel"
%	\date: when these notes were last updated, e.g. \today (optional)
%	\email: the author's email, e.g. "seuss_on_the_loose@hotmail.com"
%	\thanks: an acknowledgement, e.g. "Thanks to Emily Dickinson for fixing a few typos" (optional)
% Using the following commands, the user will be able to simply write
% \institution{UT Austin}
% \coursenum{M364C}
% and so on.
\newcommand*{\institution}[1]{\newcommand{\@institution}{#1}}
\newcommand*{\coursenum}[1]{\newcommand{\@coursenum}{#1}}
\newcommand*{\coursename}[1]{\newcommand{\@coursename}{#1}}
\newcommand*{\semester}[1]{\newcommand{\@semester}{#1}}
\newcommand*{\topic}[1]{\newcommand{\@topic}{#1}}
\renewcommand*{\email}[1]{\newcommand{\@email}{#1}}
\renewcommand{\@thanks}{\@empty}
\renewcommand{\thanks}[1]{\renewcommand{\@thanks}{#1}}
%
\let\@oldauthor\author
\newcommand*{\d@e}{\@empty}
\renewcommand*{\author}[1]{\renewcommand{\@author}{#1}}
\renewcommand*{\date}[1]{\renewcommand{\d@e}{#1}}
%
% These things should be defined after the user sets everything up in the preamble
\AtBeginDocument{
	% Email with a MAILTO: link
	\newcommand{\linkedemail}{\href{mailto:\@email?subject=\@coursenum\%20\@topic}{\texttt{\@email}}}
	%
	% Front Stuff: standard blurb at the beginning of a document, ToC, etc.
	% Once everything is set up, the user only needs to call \frontstuff.
	%
	% Format title based on amsart's own title command
    \title{\@coursenum{}: \MakeUppercase{\@topic}}
	\@oldauthor{\@author\\\d@e}
	%
	\newcommand{\@blurb}{
		{\small This was prepared for \@institution’s \@coursenum{} (\@coursename) in \@semester. Please send questions, comments, complaints, and corrections to \linkedemail. Any mistakes are my own. \@thanks}
	}
	%
	\newcommand{\frontstuff}{
		\maketitle
		\@blurb
		\tableofcontents
	}
	%
	% Fancy headers
	% TODO this doesn't work like I want it to, and I don't know how to fix it.
	\renewcommand{\sectionmark}[1]{\markleft\thesection. #1}
	%
	\fancyhf{}
	\fancyhead[RO,LE]{\small\thepage}
	\fancyhead[LO]{\small\itshape\nouppercase{\leftmark}}
	\fancyhead[RE]{\small\itshape \@coursenum{}: \@topic}
	\setlength{\headheight}{11.0pt}
	\pagestyle{fancy}
}
