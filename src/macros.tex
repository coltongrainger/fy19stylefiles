#include "xy_macros.tex"
#include "letters.tex"
#include "theorems.tex"

% sourced from Arun Debray <coltongrainger, 2019-04-14> %
%
% Unfortunately, this has to be placed before loading the cleveref package, so it's
% in packages.tex.
% \numberwithin{equation}{section}
%
% This could be considered a stylistic choice, but works much better with how I use
% subsections as subject changes in a lecture.
\setcounter{tocdepth}{1}
%
% These can be changed downstream
\makeindex % https://en.wikibooks.org/wiki/LaTeX/Indexing
\newcommand{\term}[1]{\emph{#1}\index{#1}} % e.g. "The \term{trace} is defined to be..."
\newcommand{\latin}{\textit} % e.g. \latin{per se}, \latin{mutatis mutandis}
%
% Often, I find myself making a theorem, definition, etc. that's purely a combination
% of statements, either bulleted or numbered. In that case, using \hfill typesets each
% item uniformly. It would probably be best to make this into a custom enumitem
% environment, especially if I want to refer to specific items (e.g. 12.2.1 inside
% Theorem 12.2).
% Usage: \begin{comp}[eggs]{thm}{enumerate}
%			\item
% 		 \end{comp}
% sets up an enumerate environment inside a theorem, and labels it "eggs."
\NewDocumentEnvironment{comp}{omm}{%
	\csname #2\endcsname\hfill
	% check for optional argument
	\IfNoValueF{#1}{
		\label{#1}
	}
	\csname #3\endcsname
}{
	\csname end#3\endcsname
	\csname end#2\endcsname
}
%
% since I find myself using \renewcommand{...}{\operatorname{...}} a lot.
\newcommand{\RenewMathOperator}[2]{\renewcommand{#1}{\operatorname{#2}}}
%
% other shortcuts I use for live-TeXing
\newcommand{\vp}{\varphi}
\newcommand{\e}{\varepsilon}
\newcommand{\inj}{\hookrightarrow}
\newcommand{\surj}{\twoheadrightarrow}
\newcommand{\id}{\mathrm{id}}
\newcommand{\pt}{\mathrm{pt}}
\newcommand{\many}[2][\dotsb]{#2 #1 #2} % optional argument for kind of dots
\newcommand{\TFAE}{The following are equivalent}
\newcommand{\TODO}{\textcolor{red}{TODO}}
%
% use a longer \mapsto in display math
\let\shortmapsto\mapsto
\renewcommand{\mapsto}{\mathchoice{\longmapsto}{\shortmapsto}{\shortmapsto}{\shortmapsto}}
%
% This allows \paren{...} to replace \left(...\right) (and similarly for \bkt). For
% \ang, \set, \abs, and \norm, I find myself using autoexpansion less often.
% This code, along with some other shortcuts, is duplicated in my problem set template;
% perhaps I should have them include a common file?
\providecommand{\paren}[1]{\qty( #1 )}
\providecommand{\ang}[1]{\left\langle #1 \right\rangle}
\renewcommand{\abs}[1]{\qty| #1 |} % redefinition from physics.sty
% \norm (defined in physics package)
\providecommand{\bkt}[1]{\qty[ #1 ]}
\providecommand{\set}[1]{\qty{ #1 }}
%
% Now, for a bunch of field-specific commands.
% TODO: document
% TODO: starred command, perhaps as http://tex.stackexchange.com/a/4388/
\newcommand{\newoperator}[1]{\expandafter\DeclareMathOperator\csname #1\endcsname{\operatorname{#1}}}
%
% Algebra
\newoperator{Ann}
\newoperator{Aut}
\newoperator{Cliff}
\newoperator{chr}
\newoperator{coker}
% det (defined in amsmath)
\newoperator{End}
\newoperator{Ext}
\AtBeginDocument{ % some packages redefine this
	\RenewMathOperator{\ev}{ev} 
}
\newoperator{Frac}
\newoperator{Gal}
\newoperator{Hom}
% ker (defined in amsmath)
\AtBeginDocument{ % some packages redefine this
	\RenewMathOperator{\Im}{Im} % also complex analysis
}
\newoperator{Mat}
\renewcommand{\op}{^{\cat{op}}} % for the opposite of a category
\newcommand{\rop}{^{\mathrm{op}}} % for the opposite of an algebra or monoid
\AtBeginDocument{ 
	\RenewMathOperator{\rank}{rank}
}
\newoperator{sign}
% \newoperator{span} 
\newoperator{Stab}
\newoperator{Syl}
\newoperator{Fix}
\newoperator{Orb}
\newoperator{core}
\newoperator{Norm}
\newoperator{Sym}
\newoperator{Tor}
% \newoperator{tr} (defined in physics)
\newcommand{\bl}{\text{--}}
\newcommand{\nrel}{\triangleleft}
%
% Algebraic Geometry
\newoperator{Proj}
\newoperator{QCoh}
\newoperator{res} % also complex analysis
\newoperator{Spec}

% Algebraic Topology
\newcommand{\Gr}{\mathrm{Gr}} % Grassmannian -- perhaps move to letters.tex
\newcommand{\hdr}{H_{\mathrm{dR}}}
\newcommand{\KO}{\mathit{KO}}
\newcommand{\KU}{\mathit{KU}}
\newcommand{\KR}{\mathit{KR}}
\newcommand{\ko}{\mathit{ko}}
\newcommand{\ku}{\mathit{ku}}
\newcommand{\Sq}{\mathrm{Sq}}
\newcommand{\wH}{\widetilde H}
\DeclareMathOperator*{\colim}{colim} % colimit notation in homotopy theory
\DeclareMathOperator*{\holim}{holim} % homotopy limit
\DeclareMathOperator*{\hocolim}{hocolim} % homotopy colimit
% some Thom spectra
% TODO: maybe a command to define a whole bunch at once
% or a command for Lie groups: define \G, \MG, \MTG
%
% calling \NewThomSpectrum{G} defines \MG -> \mathit{MG}
% calling \NewMTSpectrum{G} defines \MTG -> \mathit{MTG}
\newcommand{\NewThomSpectrum}[1]{\expandafter\newcommand\csname M#1\endcsname{\mathit{M#1}}}
\newcommand{\NewMTSpectrum}[1]{\expandafter\newcommand\csname MT#1\endcsname{\mathit{MT#1}}}
\newcommand{\BothThomSpectra}[1]{\NewThomSpectrum{#1}\NewMTSpectrum{#1}}
\BothThomSpectra{O}
\BothThomSpectra{SO}
\BothThomSpectra{Spin}
\BothThomSpectra{Pin}
\BothThomSpectra{U} % TODO: I haven't tested these
%\newcommand{\MR}{\mathit{MR}}
%
%
% Complex analysis
% note: Re and Im changes are technically style changes, but almost everyone uses this notation
\AtBeginDocument{
	\RenewMathOperator{\Re}{Re}
}
\newcommand{\delbar}{\overline\partial}
%
% Topology
% \dim (defined in amsart)
\newoperator{codim}
\newoperator{crit}
% \newoperator{curl} (defined in physics)
\newoperator{Diff}
\RenewMathOperator{\div}{div}
\newoperator{ind}
\newoperator{supp}
\newoperator{Cl}
\newoperator{Int}
%
% TQFT
\newcommand{\Bord}{\cat{Bord}}
\newcommand{\TQFT}{\cat{TQFT}}
\newcommand{\fr}{^{\mathrm{fr}}}
%
% to be continued (e.g. a good argmin and argmax)
